\paragraph{}
In this subsection we will talk about our graduation project main idea
\textbf{Emotion recognition} and its value to be added to the market today.
\newline
The idea is to make the machine recognize how customer feel about your product "whatever what is this product" to gain precise feedback, you expect employees to have high levels of emotional intelligence when interacting with that. Now, thanks to advances in Deep Learning, you’ll soon expect your software to do the same, this helps for better user experience.
\bigbreak
Companies have also been taking advantage of emotion recognition to drive business outcomes. For the upcoming release of Toy Story 5, Disney plans to use facial recognition to judge the emotional responses of the audience. Apple even released a new feature on the iPhone X called Animoji, where you can get a computer simulated emoji to mimic your facial expressions. It’s not so far off to assume they’ll use those capabilities in other applications soon.
\bigbreak
we tried to develop a model that help us to do so lying on some technologies to extract features from the images, like Histogram of Oriented Gradients (HOG), facial landmarks and some other preprocessing for the image, that will be discussed in the next few pages in this chapter.\newline
we have checked multiple other project working on same topic but didn't find a satisfactory answer as to what is the best way to detect emotions in real time? what best performance that can be reached and how? the compromise between real time operation and accuracy?.
for this reason we started this project with the purpose of answering these questions.