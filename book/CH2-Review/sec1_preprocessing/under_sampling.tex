Data is called imbalanced if the output class values aren't evenly represented, for example if we study data with output feature having 2 possible values, either 0 or 1, the value 0 comes 80\% of the time while the value 1 only comes 20\% of the time, if model trained on this data would be biased towards the value 0 which appears to be the most common outcome.\newline
the solution for this is to balance the data by :
\begin{enumerate}
	\item UnderSampling.
	\item Data Augmentation
\end{enumerate}
\subsubsection{Under Sampling}
it's a technique we can use when the data distribution is imbalanced, by removing some instances from the large class(s) until they are close enough to the smaller classes, this way we can decrease the bias of our model towards the large classes.\newline 

Under sampling should solve the problem with bias in the trained model, but at the cost of decreasing the size of the dataset hence decreasing what the model learns, so it can decrease the accuracy as well, if the biggest problem for your model is bias then under sampling can solve it, however if the biggest problem is in learning itself then under sampling shall make it worse.

