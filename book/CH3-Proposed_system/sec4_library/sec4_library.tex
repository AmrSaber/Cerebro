\subsection{architecture}
We represent the project to be a library so that if any other project needs to use any functionality we support they can do so easily by just downloading and importing our library.
\bigbreak

The library can be used by the higher level APIs or 
lower level methods and its structure is as follows :
\begin{itemize}
	\item Emotion Recognition Interface 
	\begin{itemize}
		\item Real Time Streaming
		\item Video
		\item Single Image
	\end{itemize}
	\item Lower Level APIs
	\begin{itemize}
		\item Model API
		\item Feature Extraction API
		\item Image Processing API
	\end{itemize}
\end{itemize}

\newpage

\begin{center}
	\subsection{Emotion Recognition Interface}
	this is an interface provides to user three modules may be needed as a block in their applications without getting deeper and deeper in the details of the model or the feature extraction methods or even in the preprocessing we did.   
\end{center}
\newpage

\subsubsection{Real Time Streaming}
The Real Time Stream interface is developed in a way that works with threads so that the stream running smoothly with high efficiency with no noticed delay in processing the frames to recognize the emotions 
\bigbreak 
\noindent\textbf{\textit{Input}}:
\begin{itemize}
	\item \textbf{skip}:\newline
	The skip value is an integer represents the number of frames will be skipped without processing defaults to 40.
\end{itemize}
\noindent\textbf{\textit{Output}}:
\begin{itemize}
	\item there is no return for that function it shows the stream with recognized faces emotions once it is called.
\end{itemize}
\noindent\textbf{\textit{To use the stream function}}:
% indentation of verbatim must be like that please don't change it
\begin{itemize}
	\item import the module:
	\begin{verbatim}
	from interface Import video-stream as vs
	\end{verbatim}
	\item call it:
	\begin{verbatim}
	vs.detect_stream_emotions(skip)
	\end{verbatim}
\end{itemize}

\newpage

\subsubsection{Video}

\newline \textbf{\textit{Input}}: 
\begin{itemize}
	\item  \textbf{video\_path}: the video which supposed to be processed.
	\item  \textbf{output\_path}: output video path, should contain output video name with its extension (must be mp4).
	\item  \textbf{skip}: how many frames will be skipped without emotion recognition , be careful by increasing this parameter the output would be less accurate but the processing would be faster, it defaults to 50 which mean detect emotion every 2 seconds in normal video speed.
	\item \textbf{detector\_type}: the default is dlib detector but it can be controlled to be haar or lbp classifiers.
	\item  \textbf{verbose}: that bool is set to True to give you information about processing operation, it defualts to False.
\end{itemize}
\textbf{\textit{Output}}: it doesn't return anything as its output directly saved to the output path.
\newline\textbf{\textit{ Rule}}: it divides the video into frames and audio, applies analysis on it then gathers them again.
\newline 
\noindent \textbf{\textit{To use the Single Image Interface }}:
\begin{itemize}
	\item import the module:
	\begin{verbatim}
	from interface Import process_video as pv
	\end{verbatim}
	\item call it:
	\begin{verbatim}
	pv.detect_video_emotion(video_path, output_path, skip,detector_type, verbose)
	\end{verbatim}
\end{itemize}


\newpage
\subsubsection{Single Image}

The Single Image Interface process only one image.
\newline
\noindent\textbf{\textit{Input}}:
\begin{itemize}
	\item \textbf{image}:\newline
	image of any size to recognize emotions of faces in it with type [Numpy.ndarray].
	\item \textbf{detector\_type}: the default is dlib detector but it can be controlled to be haar or lbp classifiers.
\end{itemize}
\noindent\textbf{\textit{Output}}:
\begin{itemize}
	\item \textbf{image}:\newline
	return image [Numpy.ndarray] with emotions recognized 
\end{itemize}

\noindent \textbf{\textit{To use the Single Image Interface }}:
\begin{itemize}
	\item import the module:
	\begin{verbatim}
	from interface Import process_image as pi
	\end{verbatim}
	\item call it:
	\begin{verbatim}
	pi.mark_faces_emotions(image)
	\end{verbatim}
\end{itemize}

\newpage

\begin{center}
	\subsection{Lower Level APIs}
	This is a more flexible option we provide to users of our library, by which you can use the core methods we rely upon for building our interface.
\end{center}

\newpage 

\subsubsection{Model API}
It provide you two methods with more options than interface.\newline
but first we need to create object of our model.
\bigbreak

\textbf{Model Creation}


\begin{itemize}
	\item import the module:
	\begin{verbatim}
	from model.emotions_model import *
	\end{verbatim}
	\item create new object:
	\begin{verbatim}
	model = EmotionsModel(
	verbose=False,
	create_new=False,
	use_hog=None,
	use_cnn=None,
	use_lm=None,
	emotions=None)
	\end{verbatim}
	
	
	\noindent\textbf{\textit{Input}}:
	\begin{itemize}
		\item verbose: \newline
		that bool is set to True to give you information about training, testing and feature extraction, it defualts to False.
		\item create\_new: \newline
		bool value set to be True if you want to delete the saved model and retrain the model 
		\item model specs:\newline
		must be assigned if \textbf{create\_new = True} OR there is no model saved.
		\newline
		\textbf{\textit{Note}}: if assigned without creating new model or there is file saved the model will neglect the values assigned
		\begin{itemize}
			\item use\_hog: \newline
			bool value set to True to use Histogram of Oriented Gradient in model training or False if needn't
			\item use\_cnn: \newline
			bool value set to True to use Convolutional Neural Network in model training or False if needn't
			\item use\_lm: \newline
			bool value set to True to use landmarks in model training or False if needn't
			\item emotions: \newline
			list of emotions as strings to map each index to one emotion 
			
		\end{itemize}
		
	\end{itemize}
\end{itemize}


\newpage



\noindent\textbf{1- predict(faces, prob\_emotion=False)}:
\paragraph{It can optimize the code by working on batches if your usage doesn't need to process image by image.\newline}
\bigbreak
\noindent\textbf{\textit{Input}}:
\begin{itemize}
	\item \textbf{list of images}:\newline
	it takes a list of [Numpy.ndarray] images and predict the emotion of each image.
	\item \textbf{prob\_emotion}\textit{(optional)}:\newline
	it's value tells if we want the method to return the probability of each emotion in a numpy array at the index corresponding to the image in the input list. 
\end{itemize}
\noindent\textbf{\textit{Output}}:
\begin{itemize}
	\item \textbf{if prob\_emotion=False:}:\newline
	will return list of emotions each emotion is a string at the index of each image in the input list.
	\item \textbf{if prob\_emotion=True:}:\newline
	will return list of Numpy arrays each array consists of the probability of emotions.
\end{itemize}


\textbf{\textit{To use the predict method}}:
\bigbreak
from model created in section 1.3.1

\begin{itemize}
	\item call method:
	\begin{verbatim}
	model.predict(faces, prob_emotion=False)
	\end{verbatim}
\end{itemize}

\newpage

\noindent\textbf{2- predict\_with\_vote(faces)}:
\paragraph{this method helps in recognizing the accurate emotion of one face by making a vote of ten sequential frames and get the most appearant emotion. \newline}

\noindent\textbf{\textit{Input}}:
\begin{itemize}
	\item \textbf{2D list }:\newline
	each list in it is 1D list for each person. each element in that list is an image the method evaluate the image and votes for the most emotion appears in the array then return that emotion.
\end{itemize}
\noindent\textbf{\textit{Output}}:
\begin{itemize}
	\item \textbf{list}:\newline
	will return 1D list with winner emotion in the index corresponding to the list it came from.
\end{itemize}

\textbf{\textit{To use the predict\_with\_vote method}}:
\bigbreak
from model created in section 1.3.1

\begin{itemize}
	\item call method:
	\begin{verbatim}
	model.predict_with_vote(faces)
	\end{verbatim}
\end{itemize}