\paragraph{\underline{laplacian}} \mbox{} \\
Where: Cerebro.image.enhancement.filters.laplacian.
\subparagraph{Parameters}
\begin{changemargin}{1cm}{0cm}
	\textbf{Img}: [numpy.ndarray] the source image to be sharpened.
\end{changemargin}

\subparagraph{Returns}
\begin{changemargin}{1cm}{0cm}
	Image [numpy.ndarray] the sharpened image.
\end{changemargin}

\subparagraph{Operation}
\begin{changemargin}{1cm}{0cm}
	Sharpen the input image and return the sharpened image.
\end{changemargin}

%\hrulefill
\newpage
%---------------------------------------------------------------------------

\paragraph{\underline{median}} \mbox{} \\
Where: Cerebro.image.enhancement.filters.median.
\subparagraph{Parameters}
\begin{changemargin}{0.5cm}{0cm}
	\begin{enumerate}[noitemsep,nolistsep]
		\item \textbf{Img}: [numpy.ndarray] the source image to be denoised
		\item \textbf{(Optional)} \textbf{size}: [int] the filter size and it must be odd and greater than 1. Its default value is 3.
	\end{enumerate}
\end{changemargin}

\subparagraph{Returns}
\begin{changemargin}{1cm}{0cm}
	Image [numpy.ndarray] a filtered image.
\end{changemargin}

\subparagraph{Operation}
\begin{changemargin}{1cm}{0cm}
 	Remove salt and pepper noise from the input image and return the filtered image.
\end{changemargin}

%\hrulefill
%---------------------------------------------------------------------------
 	
\paragraph{\underline{fastNLMeans}} \mbox{} \\
Where: Cerebro.image.enhancement.filters.fastNLMeans.
\subparagraph{Parameters}
\begin{changemargin}{0.5cm}{0cm}
	\begin{enumerate} 
		\item \textbf{Img}: [numpy.ndarray] the source image to be filtered.
		\item \textbf{(Optional)} \textbf{h}: [int] regulating filter strength. Big h value perfectly removes noise but also removes image details, smaller h value preserves details but also preserves some noise. Its default value is 15.
		\item \textbf{(Optional)} \textbf{templateWindowSize}: [int] size in pixels of the template patch that is used to compute weights, should be odd. \newline Its default value is 7.
		\item \textbf{(Optional)} \textbf{searchWindowSize}: [int] size in pixels of the window that is used to compute weighted average for given pixel, Should be odd. \newline Its default value is 21.
	\end{enumerate}
\end{changemargin}

\subparagraph{Returns}
\begin{changemargin}{1cm}{0cm}
	Image [numpy.ndarray] the filtered image.
\end{changemargin}

\subparagraph{Operation}
\begin{changemargin}{1cm}{0cm}
	Remove gaussian noise from the input image and return the filtered image.
\end{changemargin}
%\hrulefill
%---------------------------------------------------------------------------

\paragraph{\underline{align\_faces}} \mbox{} \\
Where: Cerebro.image.enhancement.alignment.align\_faces
\subparagraph{Parameters}
\begin{changemargin}{1cm}{0cm}
	\textbf{img}: [numpy.ndarray] the source image (detected face) to align it.
\end{changemargin}

\subparagraph{Returns}
\begin{changemargin}{1cm}{0cm}
	Image [numpy.ndarray] the aligned image.
\end{changemargin}

\subparagraph{Operation}
\begin{changemargin}{1cm}{0cm}
	Align (Scale and Rotate) the input face (where eyes become horizontal) and return the aligned face.
\end{changemargin}
\hrulefill
%---------------------------------------------------------------------------

\paragraph{\underline{faceTracking}} \mbox{} \\
Where: Cerebro.image.FaceTracking.faceTracking
\subparagraph{Parameters}
\begin{changemargin}{0.5cm}{0cm}
	\begin{enumerate}[noitemsep,nolistsep]
		\item \textbf{frames}: list where each element represents particular frame [numpy.ndarray] and detected persons data in it [list of tuples each tuple contains face coordinated [top-lef(x, y), bottom-right(x, y)] in this frame].
		\item \textbf{framesNumber}: [int] frames number.
	\end{enumerate}
\end{changemargin}

\subparagraph{Returns}
\begin{changemargin}{1cm}{0cm}
	\begin{enumerate} 
		\item faces [list] faces of each person, each element is [list] faces of the same person in each frame, each element in the faces list is [numpy.ndarray] face.
		\item cords [list] faces coordinates of each person, each element is [list] faces coordinates of the same person in each frame, each element is [tuple] represents faces coordinates [top-lef(x, y), bottom-right(x, y)].
	\end{enumerate}
\end{changemargin}

\subparagraph{Operations}
\begin{changemargin}{1cm}{0cm}
	Track faces in the sequence of the input frames and return tracked faces and their coordinates.
\end{changemargin}